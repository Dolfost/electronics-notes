\documentclass[../../document]{subfiles}

\begin{document}
\graphicspath{{images/}}

\chapter{Electronic Circuit Components}
\section{Basic Circuit Components}
\subsection{Wires, Cables, and Connectors}
Wires and cables provide low-resistance pathways for electric currents. Most
electrical wires are made from copper or silver and typically are protected
by an insulating coating of plastic, rubber, or lacquer. Cables consist of a
number of individually insulated wires bound together to form a
multiconductor transmission line. Connectors, such as plugs, jacks, and
adapters, are used as mating fasteners to join wires and cable with other
electrical devices. \cite{practical_electronics}

\subsubsection{Skin Effect}
The movement of electrons toward the surface of a wire under high-frequency
conditions is called the \emph{skin effect}. At low frequencies, the skin effect does
not have a large effect on the conductivity (or resistance) of the wire.
However, as the frequency increases, the resistance of the wire may become an
influential factor. \cite{practical_electronics}

One thing that can be done to reduce the resistance caused by skin effects is
to use stranded wire --- the combined surface area of all the individual wires
within the conductor is greater than the surface area for a solid-core wire
of the same diameter. \cite{practical_electronics}

\subsection{Batteries}
A battery is made up of a number of cells. Each cell contains a positive
terminal, or \emph{cathode}, and a negative terminal, or \emph{anode}. (Note
that most other devices treat \emph{anodes} as positive terminals and
\emph{cathodes} as negative terminals.) \cite{practical_electronics}

Batteries are classified into 2 types:
\begin{description}
	\item[Primary] Batteries of this type cannot be recharged.
	\item[Secondary] Batteries of this type can be recharged and reused.
\end{description}

\begin{sidewaystable}
	\begin{center}
		\small
		\begin{tabular}{p{.09\textheight}p{.07\textheight}p{.06\textheight}p{.07\textheight}p{.12\textheight}p{.05\textheight}p{.22\textheight}p{.2\textheight}}
			\toprule 
			Type\newline (Chemistry)  & Common\newline names & Voltage
																& Internal\newline resistance & Maximum\newline
			discharge rate & Cost & Pros and Cons & Typical applications\\
			\midrule
			Carbon-zinc & Standard-duty & 1.5 & Medium & Medium & Low & Low cost,
			various sizes, but terminal voltage drops steadily during cell life &
			Radios, toys, and general-purpose electrical equipment\\
			\midrule
			Zinc-chloride & Heavy-duty & 1.5 & Low & Medium to high & Low to medium &
			Low cost at higher discharge rates and at lower temperature; terminal
			voltage still drops & Motor-driven  portable devices, clocks, ­ remote
			controls\\
			\midrule
			Alkaline zinc-manganese dioxide & Alkaline & 1.5 & Very low & High &
			Medium to high & Better for high continuous or pulsed loads and at low
			temperatures, but terminal voltage drops & Photoflash units, battery
			shavers, digital cameras, handheld transceivers, portable CD players,
			etc.\\
			\midrule
			Lithium-manganese dioxide & Lithium & 3.0 & Low & Medium to high & High &
			High energy density, very low self-discharge rate (excellent shelf-life),
			good temperature tolerance & Watches, calculators, cameras (digital and
			film), DMMs, and other test instruments\\
			\midrule
			Zinc-mercuric oxide & Mercury cell & 1.35 & Low & Low & High & High
			energy density (compact), very flat discharge curve, good at higher
			temperatures & Calculators, pagers hearing aids, watches, test intruments\\
			\midrule
			Zinc-silver oxide & Silver oxide cell & 1.5 & Low & Low & High & Very
			high energy density (very compact), very flat discharge curve,
			reasonable at lower temperatures & Calculators, pagers, hearing aids,
			watches test instruments\\
			\midrule
			Zinc-oxygen & Zinc air cell & 1.45 & Medium & Low & Medium & High energy
			density, very lightweight, flat discharge curve, but must have access to
			air & Hearing aids and pagers\\
			\bottomrule
		\end{tabular}
	\end{center}
	\caption{Primary Battery Comparison \cite[p. 278]{practical_electronics}}
\end{sidewaystable}

\begin{sidewaystable}
	\begin{center}
		\small
		\begin{tabular}{p{.06\textheight}p{.06\textheight}p{.05\textheight}p{.05\textheight}p{.06\textheight}p{.09\textheight}p{.05\textheight}p{.22\textheight}p{.2\textheight}}
			\toprule 
			Type & Voltage (\(\approx\)) & Energy density
			(\unit{\watt\per\kilogram}) & Cycle life & Charge time & Max. discharge
			rate & Cost & Pros and Cons & Typical applications\\
			\midrule
			Sealed\newline led-acid & 2.0 & Low (30) & Long (shallow cycles) &
			8–1\unit{\hour} & Medium\newline (0.2C) & Low & Low cost, low
			self-discharge, happy float charging, but prefers shallow charging &
			Emergency lighting, alarm systems solar power systems, wheelchairs,
			etc.\\
			\midrule
			Rechargeable alkaline-manganese & 1.5 & High (75 initial) & Short to
			medium & 2–6\unit{\hour} (pulsed) & Medium (0.3 C) & Low & Low cost, low
			self-discharge, prefer shallow cycling, no memory effect but short cycle
			life & Portable emergency lighting, toys, portable radios, CD players,
			test instruments, etc.\\
			\midrule
			NiCad & 1.2 & Medium (40–60) & Long (deep cycles) & 14–16\unit{\hour}
			(0.1C) or <2\unit{\hour} with care (1C) & High (>2C) & Medium & Prefer
			deep cycling, good pulse capacity, but have memory effect, fairly high
			self-discharge rate, environmentally unfriendly & Portable tools and
			appliances, model cars and boats, data loggers, camcorders, portable
			transceivers, and test equipment\\
			\midrule
			NiMH & 1.2 & High (60–80) & Medium & 2–4\unit{\hour} & Medium (0.2–0.5C)
					 & Medium & Very compact energy source, but have some memory effect,
			high self-discharge rate & Remote control ­ vehicles, cordless ­ mobile
			phones, personal DVD and CD players, power tools\\ 
			\midrule
			NiZn & 1.65 & High
			(>170) & Medium to high & 1–2\unit{\hour} & --- & Medium & Low cost,
			environmentally green, twice ­ energy density of Ni­ Ca & Exceptional
			performance, no memory, long shelf-life\\
			\midrule
			NiFe & 1.4 & High (>200)  & Extr. long & Long & --- & Low & High
			cycle life, incredibly long life up to 80 years, environmentally friendly
					 & Forklifts and other, similar SLA-like applications, but where
			longevity is important\\ 
			\midrule
			Li-ion/LiPo & 3.6 & Very high (>100) & Medium &
			3–4\unit{\hour} (1–0.03C) &  Med/high (<1C) & High & Very compact, low
			maintenance, low self-discharge, but needs great care with charging &
			Compact cell phones and notebook PCs, digital cameras, and similar very
			small ­ portable devie\\
			\bottomrule
		\end{tabular}
	\end{center}
	\caption{Rechargeable Battery Comparison \cite[p. 284]{practical_electronics}}
\end{sidewaystable}

\subsubsection{Supercapacitor}
Supercapacitor is a cross between a capacitor and a battery. It resembles a
regular capacitor, but uses special electrodes and some electrolytes. There are
three kinds of electrode material found in a supercapacitor:
high-surface-area-activated carbons, metal oxide, and conducting polymers. The
one using high-surface-area-activated carbons is the most economical to
manufacture. \cite{practical_electronics}

Limitations include an inability to use the full energy spectrum --- depending
on the application, not all energy is available. A supercapacitor has low
energy density, typically holding \(1/5\)  to \(1/10\) the energy of an
electrochemical battery. Cells have low voltages --- serial connections are
needed to obtain higher voltages. Voltage balancing is required if more than
three capacitors are connected in series. Furthermore, the self-discharge is
considerably higher than that of an electrochemical battery. 
\cite{practical_electronics}

Advantages include a virtually unlimited life cycle --- supercapacitors are not
subject to the wear and aging experienced by electrochemical batteries. Also,
low impedance can enhance pulsed current demands on a battery when placed in
parallel with the battery. Supercapacitors experience rapid charging --- with
low-impedance versions reaching full charge within seconds. The charge method
is simple --- the voltage-limiting circuit compensates for self-discharge.
\cite{practical_electronics}

\subsubsection{Battery Capacity}
Batteries are given a capacity rating that indicates how much electrical energy
they are capable of delivering over a period of time. The capacity rating is
specified in terms of ampere-hours (\unit{\ampere\hour}) and millampere-hours
(\unit{\milli\ampere\hour}). \cite{practical_electronics}

\subsubsection{C Rating}
The charge and discharge currents of a battery are measured in capacity rating
or C rating. The capacity represents the efficiency of a battery to store
energy and its ability to transfer this energy to a load. Most portable
batteries, with the exception of lead-acid, are rated at 1C. A discharge rate
of 1C draws a current equal to the rated capacity that takes one hour. For
example, a battery rated at 1000 \unit{\milli\ampere\hour} provides 1000
\unit{\milli\ampere} for 1 hour if discharged at 1C rate. The same battery at
0.5C provides 500 \unit{\milli\ampere} for 2 hours. At 2C, the same battery
delivers 2000 \unit{\milli\ampere} for 30 minutes. 1C is often referred to as a
1-hour discharge; 0.5C would be 2 hours, and 0.1C would be a 10-hour discharge.
The discrepancy in C rates between different batteries is largely dependent on
the internal resistance. \cite{practical_electronics}

\subsubsection{Internal Voltage Drop}
\begin{wraptable}[23]{r}{.5\linewidth}
	\begin{center}
		\begin{tabular}{ll}
			\toprule 
			Battery & Resistance\\
			\midrule
			9\unit{\volt} zinc carbon & 35\unit{\ohm}\\
			9\unit{\volt} lithium & 16 to 18\unit{\ohm}\\
			9\unit{\volt} alkaline & 1 to 2\unit{\ohm}\\
			AA alkaline & 0.15\unit{\ohm} (0.30\unit{\ohm} at 50\%)\\
			AA NiMH & 0.02\unit{\ohm} (0.04\unit{\ohm} at 50\%)\\
			D alkaline & 0.1\unit{\ohm}\\
			D NiCad & 0.009\unit{\ohm}\\
			D SLA & 0.006\unit{\ohm}\\
			AC13 zinc air & 5\unit{\ohm}\\
			76 silver & 10\unit{\ohm}\\
			675 mercury & 10\unit{\ohm}\\
			\bottomrule
		\end{tabular}
	\end{center}
	\caption{Typical internal resistance for various batteries \cite[p. 290]{practical_electronics}}
\end{wraptable}
Batteries have an internal resistance that is a result of the imperfect
conducting elements that make up the battery (resistance in electrodes and
electrolytes). Though the internal resistance may appear low (around
0.1\unit{\ohm} for an AA alkaline battery, or 1 to 2\unit{\ohm} for a
9\unit{\volt} alkaline battery), it can cause a noticeable drop in output
voltage if a low-resistance (high-current) load is attached to it. Without a
load, we can measure the open-circuit voltage of a battery. This voltage is
essentially equal to the battery's rated nominal voltage --- the voltmeter has
such a high input resistance that it draws practically no current, so there is
no appreciable voltage drop. However, if we attach a load to the battery, the
output terminal voltage of the battery drops. By treating the internal
resistance Rin and the load resistance Rload as a voltage divider, you can
calculate the true output voltage present across the load.
\cite{practical_electronics}

Batteries with large internal resistances show poor performance in supplying
high current pulses. Internal resistance also increases as the battery
discharges. \cite{practical_electronics}

\subsection{Switches}
A switch is a mechanical device that interrupts or diverts electric current
flow within a circuit. \cite{practical_electronics}

A switch is characterized by its number of \emph{poles} (\(P\)) and by its
number of \emph{throws} (\(T\)). A pole represents a contact, meanwhile throw
represents the particular contact-to-contact connection. When specifying the
number of poles and the number of throws, a convention must be followed: When
the number of poles or number of throws equals 1, the letter S, which stands
for \enquote{single} is used. When the number of poles or number of throws
equals 2, the letter D, which stands for \enquote{double} is used. When the
number of poles or number of throws exceeds 2, integers such as 3, 4, or 5 are
used. Examples: SPST, SPDT, DPST, DPDT, DP3T, and 3P6T.
\cite{practical_electronics}

Two important features to note about switches include whether a switch has ­
momentary contact action and whether the switch has a center-off position.
Momentary-contact switches, which include mainly pushbutton switches, are used
when it is necessary to only briefly open or close a connection.
Momentary-contact switches come in either normally closed (NC) or normally open
(NO) forms. A normally closed pushbutton switch acts as a closed circuit
(passes current) when left­ untouched. A normally open pushbutton switch acts
as an open circuit (broken circuit) when left untouched. Center-off position
switches, which are seen in diverter switches, have an additional \enquote{off}
position located between the two \enquote{on} positions.
\cite{practical_electronics}
 
\subsection{Relays}
\emph{Relays} are electrically actuated switches. The three basic kinds of
relays include mechanical relays, reed relays, and solid state relays. For a
typical mechanical relay, a current sent through a coil magnet acts to pull a
flexible, spring-loaded conductive plate from one switch contact to another.
Reed relays consist of a pair of reeds (thin, flexible metal strips) that
spring together whenever a current is sent through an encapsulating wire coil.
A solid-state relay is a device that can be made to switch states by applying
external voltages across n-type and p-type semiconductive junctions. In
general, mechanical relays are designed for high currents (typically 2 to
15\unit{\ampere}) and relatively slow switching (typically 10 to
100\unit{\milli\second}). Reed relays are designed for moderate currents
(typically 500\unit{\milli\ampere} to 1\unit{\ampere} and moderately fast
switching (0.2 to 2\unit{\milli\second}). Solid-state relays, on the other
hand, come with a wide range of current ratings (a few microamps for
low-powered packages up to 100\unit{\ampere} for high-power packages) and have
extremely fast switching speeds (typically 1 to 100\unit{\nano\second}). Some
limitations of both reed relays and solid-state relays include limited
switching arrangements (type of switch section) and a tendency to become
damaged by surges in power. \cite{practical_electronics}

\subsection{Resistors}
Resistors perform two basic functions in electronics: to limit current flow and
to set voltage levels within a circuit.

\subsubsection{Voltage Rating}
This is the maximum value of dc or RMS voltage that can be imposed across a ­
resistor at specified ambient temperatures. The voltage rating is related to
the power \(V=\sqrt{P\times R}\), where \(V\) is the voltage rating (in volts),
\(P\) is the power rating (in watts), and \(R\)  is the resistance (in ohms).
\cite{practical_electronics}

\subsubsection{Tolerance}
This is expressed as the deviation (in percent) in resistance from the nominal
value, measured at 2\unit{\celsius} with no load applied. Typical resistor
tolerances are 1 percent 2 percent, 5 percent, 10 percent, and 20 percent.
Precision resistors, such as precision wirewounds, are made with tolerances as
tight as \pm 0.005 percent. \cite{practical_electronics}

\subsubsection{Power Rating}
Resistors must be operated within specified temperature limits to avoid
permanent damage to the materials. The temperature limit is defined in terms of
the maximum power, called the power rating, and a derating curve that is
provided by the resistor manufacturers. The power rating of a resistor is the
maximum power in watts that the resistor can safely dissipate as heat, usually
specified at +25\unit{\celsius}. \cite{practical_electronics}

\subsubsection{Temperature Coefficient of Resistance (TCR or TC)}
This tells the amount of resistance change that occurs when the temperature of
a resistor changes. TC values are typically expressed as parts per million
(ppm) for each degree centigrade change from some nominal temperature-usually
room temperature (25\unit{\celsius}). \cite{practical_electronics}

So a resistor with a TC of 100 ppm will change 0.1 percent in resistance over a
10\unit{\celsius} change, and will change 1 percent over a 100\unit{\celsius}
change (provided the temperature change is within the resistor’s rated
temperature range, e.g., -55 to +145\unit{\celsius}, measured at
25\unit{\celsius} room temperature). A positive TC means an increase in
resistance with increasing temperature, while a negative TC indicates a
decreasing resistance with an increase in temperature.
\cite{practical_electronics}

\end{document}
