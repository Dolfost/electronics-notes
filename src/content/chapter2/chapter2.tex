\documentclass[../../document]{subfiles}

\begin{document}
\graphicspath{{images/}}

\chapter{Electronic Circuit Components}
\section{Basic Circuit Components}
\subsection{Wires, Cables, and Connectors}
Wires and cables provide low-resistance pathways for electric currents. Most
electrical wires are made from copper or silver and typically are protected
by an insulating coating of plastic, rubber, or lacquer. Cables consist of a
number of individually insulated wires bound together to form a
multiconductor transmission line. Connectors, such as plugs, jacks, and
adapters, are used as mating fasteners to join wires and cable with other
electrical devices. \cite{practical_electronics}

\paragraph{Skin Effect}
The movement of electrons toward the surface of a wire under high-frequency
conditions is called the \emph{skin effect}. At low frequencies, the skin effect does
not have a large effect on the conductivity (or resistance) of the wire.
However, as the frequency increases, the resistance of the wire may become an
influential factor. \cite{practical_electronics}

One thing that can be done to reduce the resistance caused by skin effects is
to use stranded wire --- the combined surface area of all the individual wires
within the conductor is greater than the surface area for a solid-core wire
of the same diameter. \cite{practical_electronics}

\subsection{Batteries}
A battery is made up of a number of cells. Each cell contains a positive
terminal, or \emph{cathode}, and a negative terminal, or \emph{anode}. (Note
that most other devices treat \emph{anodes} as positive terminals and
\emph{cathodes} as negative terminals.) \cite{practical_electronics}

\end{document}
