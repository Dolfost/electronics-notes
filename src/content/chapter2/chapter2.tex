\documentclass[../../document]{subfiles}

\begin{document}
\graphicspath{{images/}}

\chapter{Electronic Circuit Components}
\section{Basic Circuit Components}
\subsection{Wires, Cables, and Connectors}
Wires and cables provide low-resistance pathways for electric currents. Most
electrical wires are made from copper or silver and typically are protected
by an insulating coating of plastic, rubber, or lacquer. Cables consist of a
number of individually insulated wires bound together to form a
multiconductor transmission line. Connectors, such as plugs, jacks, and
adapters, are used as mating fasteners to join wires and cable with other
electrical devices. \cite{practical_electronics}

\subsubsection{Skin Effect}
The movement of electrons toward the surface of a wire under high-frequency
conditions is called the \emph{skin effect}. At low frequencies, the skin effect does
not have a large effect on the conductivity (or resistance) of the wire.
However, as the frequency increases, the resistance of the wire may become an
influential factor. \cite{practical_electronics}

One thing that can be done to reduce the resistance caused by skin effects is
to use stranded wire --- the combined surface area of all the individual wires
within the conductor is greater than the surface area for a solid-core wire
of the same diameter. \cite{practical_electronics}

\subsection{Batteries}
A battery is made up of a number of cells. Each cell contains a positive
terminal, or \emph{cathode}, and a negative terminal, or \emph{anode}. (Note
that most other devices treat \emph{anodes} as positive terminals and
\emph{cathodes} as negative terminals.) \cite{practical_electronics}

Batteries are classified into 2 types:
\begin{description}
	\item[Primary] Batteries of this type cannot be recharged.
	\item[Secondary] Batteries of this type can be recharged and reused.
\end{description}

\begin{sidewaystable}
	\begin{center}
		\small
		\begin{tabular}{p{.09\textheight}p{.07\textheight}p{.06\textheight}p{.07\textheight}p{.12\textheight}p{.05\textheight}p{.22\textheight}p{.2\textheight}}
			\toprule 
			Type\newline (Chemistry)  & Common\newline names & Voltage
																& Internal\newline resistance & Maximum\newline
			discharge rate & Cost & Pros and Cons & Typical applications\\
			\midrule
			Carbon-zinc & Standard-duty & 1.5 & Medium & Medium & Low & Low cost,
			various sizes, but terminal voltage drops steadily during cell life &
			Radios, toys, and general-purpose electrical equipment\\
			\midrule
			Zinc-chloride & Heavy-duty & 1.5 & Low & Medium to high & Low to medium &
			Low cost at higher discharge rates and at lower temperature; terminal
			voltage still drops & Motor-driven  portable devices, clocks, ­ remote
			controls\\
			\midrule
			Alkaline zinc-manganese dioxide & Alkaline & 1.5 & Very low & High &
			Medium to high & Better for high continuous or pulsed loads and at low
			temperatures, but terminal voltage drops & Photoflash units, battery
			shavers, digital cameras, handheld transceivers, portable CD players,
			etc.\\
			\midrule
			Lithium-manganese dioxide & Lithium & 3.0 & Low & Medium to high & High &
			High energy density, very low self-discharge rate (excellent shelf-life),
			good temperature tolerance & Watches, calculators, cameras (digital and
			film), DMMs, and other test instruments\\
			\midrule
			Zinc-mercuric oxide & Mercury cell & 1.35 & Low & Low & High & High
			energy density (compact), very flat discharge curve, good at higher
			temperatures & Calculators, pagers hearing aids, watches, test intruments\\
			\midrule
			Zinc-silver oxide & Silver oxide cell & 1.5 & Low & Low & High & Very
			high energy density (very compact), very flat discharge curve,
			reasonable at lower temperatures & Calculators, pagers, hearing aids,
			watches test instruments\\
			\midrule
			Zinc-oxygen & Zinc air cell & 1.45 & Medium & Low & Medium & High energy
			density, very lightweight, flat discharge curve, but must have access to
			air & Hearing aids and pagers\\
			\bottomrule
		\end{tabular}
	\end{center}
	\caption{Primary Battery Comparison \cite[p. 278]{practical_electronics}}
\end{sidewaystable}

\begin{sidewaystable}
	\begin{center}
		\small
		\begin{tabular}{p{.06\textheight}p{.06\textheight}p{.05\textheight}p{.05\textheight}p{.06\textheight}p{.09\textheight}p{.05\textheight}p{.22\textheight}p{.2\textheight}}
			\toprule 
			Type & Voltage (\(\approx\)) & Energy density
			(\unit{\watt\per\kilogram}) & Cycle life & Charge time & Max. discharge
			rate & Cost & Pros and Cons & Typical applications\\
			\midrule
			Sealed\newline led-acid & 2.0 & Low (30) & Long (shallow cycles) &
			8–1\unit{\hour} & Medium\newline (0.2C) & Low & Low cost, low
			self-discharge, happy float charging, but prefers shallow charging &
			Emergency lighting, alarm systems solar power systems, wheelchairs,
			etc.\\
			\midrule
			Rechargeable alkaline-manganese & 1.5 & High (75 initial) & Short to
			medium & 2–6\unit{\hour} (pulsed) & Medium (0.3 C) & Low & Low cost, low
			self-discharge, prefer shallow cycling, no memory effect but short cycle
			life & Portable emergency lighting, toys, portable radios, CD players,
			test instruments, etc.\\
			\midrule
			NiCad & 1.2 & Medium (40–60) & Long (deep cycles) & 14–16\unit{\hour}
			(0.1C) or <2\unit{\hour} with care (1C) & High (>2C) & Medium & Prefer
			deep cycling, good pulse capacity, but have memory effect, fairly high
			self-discharge rate, environmentally unfriendly & Portable tools and
			appliances, model cars and boats, data loggers, camcorders, portable
			transceivers, and test equipment\\
			\midrule
			NiMH & 1.2 & High (60–80) & Medium & 2–4\unit{\hour} & Medium (0.2–0.5C)
					 & Medium & Very compact energy source, but have some memory effect,
			high self-discharge rate & Remote control ­ vehicles, cordless ­ mobile
			phones, personal DVD and CD players, power tools\\ 
			\midrule
			NiZn & 1.65 & High
			(>170) & Medium to high & 1–2\unit{\hour} & --- & Medium & Low cost,
			environmentally green, twice ­ energy density of Ni­ Ca & Exceptional
			performance, no memory, long shelf-life\\
			\midrule
			NiFe & 1.4 & High (>200)  & Extr. long & Long & --- & Low & High
			cycle life, incredibly long life up to 80 years, environmentally friendly
					 & Forklifts and other, similar SLA-like applications, but where
			longevity is important\\ 
			\midrule
			Li-ion/LiPo & 3.6 & Very high (>100) & Medium &
			3–4\unit{\hour} (1–0.03C) &  Med/high (<1C) & High & Very compact, low
			maintenance, low self-discharge, but needs great care with charging &
			Compact cell phones and notebook PCs, digital cameras, and similar very
			small ­ portable devie\\
			\bottomrule
		\end{tabular}
	\end{center}
	\caption{Rechargeable Battery Comparison \cite[p. 284]{practical_electronics}}
\end{sidewaystable}

\subsubsection{Supercapacitor}

\end{document}
