\documentclass[../../document]{subfiles}

\begin{document}
\graphicspath{{images/}}

\chapter{Theory}
\section{Electric current}
Electric current is the total charge that passes through some cross-sectional
area A per unit time. This cross-sectional area could represent a disk placed
in a gas, plasma, or liquid. \cite{practical_electronics}

If \(\Delta Q\) is the amount of charge passing through an area in a time
interval \(\Delta t\) , then the average current \(I_{\text{ave}}\) is defined as: \cite{practical_electronics}
\begin{gather}
	I_{\text{ave}} = \frac{\Delta Q}{\Delta t}.
\end{gather}

If the current changes with time, we define the instantaneous current I by
taking the limit as \(\Delta t\to 0\), so that the current is the instantaneous
rate at which charge passes through an area: \cite{practical_electronics}
\begin{gather}
	I = \lim_{\Delta t \to 0} \frac{\Delta Q}{\Delta t} = \frac{\mathrm d Q}{\mathrm d t}.
\end{gather}
The unit of current is coulombs per second, or \emph{ampere} (\(\unit{\ampere}\)):
\begin{gather}
	1\unit{\ampere} = 1\unit{\coulomb\per\second}.
\end{gather}
In circuit electrons are the who doing all the work and carry energy. 
\begin{gather}
	1\unit{\ampere} = -6.24\times 10^{18}\unit{electrons\per\second}.
\end{gather}

\section{Voltage}
To get electrical current to flow from one point to another, a voltage must
exist between the two points. A voltage placed across a conductor gives rise to
an \emph{electromotive force} (EMF) that is responsible for giving all free
electrons within the conductor a push. \cite{practical_electronics}

Voltage is also referred to as a potential difference or just potential ---
they all mean the same thing. Volt is defined to be \cite[p. 12]{practical_electronics}
\begin{gather}
	1\unit{\volt} = \frac{1\unit{\watt}}{1\unit{\ampere}}.
\end{gather}





\end{document}
