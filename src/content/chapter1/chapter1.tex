\documentclass[../../document]{subfiles}

\begin{document}
\graphicspath{{images/}}

\chapter{Theory}
\section{Electric current}
Electric current is the total charge that passes through some cross-sectional
area A per unit time. This cross-sectional area could represent a disk placed
in a gas, plasma, or liquid. \cite{practical_electronics}

If \(\Delta Q\) is the amount of charge passing through an area in a time
interval \(\Delta t\) , then the average current \(I_{\text{ave}}\) is defined as: \cite{practical_electronics}
\begin{gather}
	I_{\text{ave}} = \frac{\Delta Q}{\Delta t}.
\end{gather}

If the current changes with time, we define the instantaneous current I by
taking the limit as \(\Delta t\to 0\), so that the current is the instantaneous
rate at which charge passes through an area: \cite{practical_electronics}
\begin{gather}
	I = \lim_{\Delta t \to 0} \frac{\Delta Q}{\Delta t} = \frac{\mathrm d Q}{\mathrm d t}.
\end{gather}
The unit of current is coulombs per second, or \emph{ampere} (\(\unit{\ampere}\)):
\begin{gather}
	1\unit{\ampere} = 1\unit{\coulomb\per\second} = -6.24\times 10^{18}\unit{electrons\per\second}.
\end{gather}

\section{Voltage}
To get electrical current to flow from one point to another, a voltage must
exist between the two points. A voltage placed across a conductor gives rise to
an \emph{electromotive force} (EMF) that is responsible for giving all free
electrons within the conductor a push. \cite{practical_electronics}

Voltage is also referred to as a potential difference or just potential ---
they all mean the same thing. Volt is defined to be \cite[p. 12]{practical_electronics}
\begin{gather}
	1\unit{\volt} = \frac{1\unit{\watt}}{1\unit{\ampere}}.
\end{gather}

\section{Resistance, Resistivity, Conductivity}
In 1826, Georg Simon Ohm defined the \emph{resistance} as 
\begin{align}
	R \equiv \frac{V}{I}, && 1\unit{\ohm}=1\unit{\volt}/1\unit{\ampere}.
\end{align}
Now, Ohm’s law isn’t really a law, but rather an empirical statement about the
behavior of materials. There are some materials\footnote{That said, it only can
be applied to \emph{ohmic materials.}} for which Ohm’s law actually doesn’t
work. \cite{practical_electronics}

\paragraph*{Curious note about Ohm’s law}
Usually you see Ohm’s law written in the following form:
\begin{gather}
	V=I\times R.
\end{gather}
In this form it is tempting to define voltage in terms of resistance and
current. It is important to realize that \(R\) is the resistance of an ohmic
material and is independent of \(V\)  in Ohm’s law. Ohm’s law does not say
anything about voltage, rather, it defines resistance in terms of it and cannot
be applied to other areas of physics such as static electricity, because there
is no current flow. That’s not to say that you can’t apply Ohm’s law to, say,
predict what voltage must exist across a known resistance, given a measured
current. \cite[p. 24]{practical_electronics}

\subsection{Resistivity and Conductivity}
Resistivity is a property unique to the material. The \emph{resistivity}
\(\rho\) is defined as follows:
\begin{gather}
	\rho\equiv R\frac{A}{L},
\end{gather}
where \(A\) is a cross-sectional area, \(L\) is the wire length, \(R\) is an
overall wire resistance at provided length. It is measured in
\unit{\ohm\meter}. \cite{practical_electronics}

Conductivity is an opposite to resistivity
\begin{gather}
	\sigma\equiv \frac{1}{\rho},
\end{gather}
and is measured in siemens (\unit{1\per\ohm\meter}). \cite{practical_electronics}

Generally, within a certain temperature range, the resistivity for a large
number of metals obeys the following equation:
\begin{gather}
	\rho = \rho_0[1+\alpha(T-T_0)],
\end{gather}
where \(\rho\) is calculated resistivity based on reference resistivity
\(\rho_0\) and temperature \(T_0\). Alpha \(\alpha\) is an \emph{temperature
coefficient of resistivity} given in \unit{1\per\celsius}. \cite{practical_electronics}   

\section{Heat and Power}
Power is defined as 
\begin{gather}
	P=VI = V\frac{V}{R} = \frac{V^2}{R},\\
	P=VI=(IR)I=I^2R.
\end{gather}

\section{Resistors in circuits}
When two or more resistors are placed in \textbf{parallel}, the voltage across
each resistor is the same, but the current through each resistor will vary with
resistance. Also, the otal resistance of the combination will be lower than
that of the lowest resistance value present. \cite{practical_electronics}
\begin{gather}
	R_{\text{total}} = \frac{1}{\frac{1}{R_1}+\frac{1}{R_2}+\frac{1}{R_3}+\dots},\\
	R_{\text{total}} = \frac{R_1\times R_2}{R_1+R_2}\text{ -- two resistors in parallel}.
\end{gather}Meanwhile, current in paraller circuit is the sum of currents of every separate
component, voltage does not change:
\begin{gather}
	I_{\text{total}} = I_1+I_2+I_3+\dots+I_n,\\
	V_{\text{total}} = V_{\text{in}}.
\end{gather}

Total resistance of resistors in \textbf{series} is the sum of resistances:
\begin{gather}
	R_{\text{total}}=R_1+R_2+R_3+\dots+R_n.
\end{gather}
The sum of all the voltage drops across each series resistor will equal the
applied voltage across the combination, but current will be constant \cite{practical_electronics}
\begin{gather}
	V_{\text{total}}=V_1+V_2+V_3+\dots+V_n,\\
	I_{\text{total}} = I_{\text{const}}.
\end{gather}




\end{document}
